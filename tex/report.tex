\documentclass[titlepage]{jarticle}

\title{第三期JAVA課題}
\author{種市隼兵\\学籍番号:6313067}
\date{平成26年7月16日}
\begin{document}
\maketitle

\section{課題}

\section{アルゴリズムの説明}
\subsection{更なる速さを求めて}
提出が遅くなりました。なかなかコードが書く気が起きず、
だらだらと過ごしていたら、提出日当日になり、その日の内に書き上げようと思いましたが思った以上に手こずってしまい
提出する事が出来ませんでした。なので、おくれて出すからには、ちゃんと速さにこだわろうと思い、色々試行錯誤して見る事にしました。
まず、コードを書く上で、最後のsampleの処理速度を5秒以内にするということを目標にしました。
結果的におよそ一秒(1.1~1.7)程で処理できるコードを書く事ができました。


\section{プログラムの説明}
四点を動かせるようにした事以外は、先生の授業でやったことなので省こうと思います。説明としてはみなさんと同じようなかんじです。なので、ウサギの四点のところの説明を少ししようとおもいます。
\begin{verbatim}
  if (flag1) {
    double dy = e.getY()-py1;
    double dx = e.getX()-px1;
    r = Math.atan2(-dy,-dx);
    deg = r * 180 / Math.PI;
    usagi.turn(deg-deg2);
    length = Math.sqrt(dy*dy + dx*dx);
    usagi.move(length);
    repaint();
    px1 = e.getX();
    py1 = e.getY();
    px2 = e.getX();
    py2 = py2 - dy;
    px3 = px3 - dx;
    py3 = e.getY();
    px4 = px4 - dx;
    py4 = py4 - dy;
    deg2 = deg;
  }
\end{verbatim}
ソースコードをみてもらえればわかりますが、四点反応させる為にflagを四つ用意して、それぞれにflag1,flag2,flag3,flag4を用意しました。
上記のはそれのflag1の部分を抜き出してきたものです。
getY,getXメソッドで位置情報を獲得し、それを初期位置から引いたものが移動距離になるわけですね。py,pxはそれぞれの座標位置の変数です。py1,px1,py2,px2...px4,py4
とあります。
配列として作っても良かったのですが、flagの中身が各々変わっているので、配列として扱うメリットがなかったので作りませんでした。
付け足した点と言えば、それぞれの座標の更新と、dy,dxの正負符号くらいだと思います。ここが地味に一番めんどくさかったです。
残りのところは、コメントもつけているので、実際にコードを読んでみてもらえると嬉しいです。

\section{考察}
今回の授業ではマウスイベント処理を学びました。
実際に画面で動いたりすると楽しかったです。
でも、クラス定義が冗長だったりするので僕はRubyのほうが好きだなあと思いました。
考察することが余り思いつきませんでした。

\end{document}
